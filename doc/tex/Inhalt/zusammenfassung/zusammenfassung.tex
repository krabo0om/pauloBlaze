\chapter{Summary}
\label{ch:sum}
In this work a small 8-bit processor called PauloBlaze, which aims to replace the PicoBlaze, has been developed.
This goal was achieved, one processor can replace the other and neither hardware nor software code have to be changed.
Even though there are some compromises, the PauloBlaze offers useful features enabling it to compete.
The PicoBlaze, a hand optimized design, is smaller than the new alternative.
The user has to compensate an increase of up to \SI{150}{\percent}, but because of the initially low requirements, the overall resource usage increases only marginally.
It is also faster than a PauloBlaze, it varies from a \SI{16.0}{\percent} to an only \SI{4.3}{\percent} slowdown.
If the design does not require maximum speed or if fast 7-Series FPGAs are used, the difference is negligible.
\\
On the other hand the new processor provides the possibility for specific optimization or an easy implementation of new instructions, which may save many clock cycles.
Furthermore the designer is not limited to the Spartan-6, Virtex-6 or 7-Series devices, the vendor independent description can be deployed everywhere, as long as the target supports VHDL.
\\
The PauloBlaze trades a small performance penalty for a far more adaptive model, providing flexible designers a very useful tool.