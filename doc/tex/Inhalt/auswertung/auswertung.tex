\chapter{Evaluation}
\label{ch:res}
%transition
Implementing a design is only one step in the development, other important steps are verification and evaluation of the result.
%issue
%claim: schlechter aber konkurenzfähig
%dev s: veri dann eval
Even before the evaluation it was clear that a certain performance loss has to be accepted when writing pure VHDL instead of directly describing vendor specific elements, however the PauloBlaze still delivers good performance combined with an acceptable area increase.
%limit
%support
%nicht optimiert da anpassbarkeit wichtiger
Those measured values are based on a solution without any significant optimizations to keep the code readable, portable and easily changeable.
%conclusion
After all these are the project's main goals and that required a trade-off between them or area and speed.


\section{Verification}
\label{sec:verifi}
The whole PauloBlaze was simulated and later used in bigger designs to detect possible bugs.
To automate the simulation, a self testing program (cf.~\hyperref[{ch:test_prog}]{Appendix~\ref*{ch:test_prog}}) issues every data chancing instruction and checks the result afterwards.
Those checks helped to test jumps and other control flow instructions.
After the core passed the simulations, it was deployed into the PicoBlaze-Library \cite{picoLib} and performed well on Xilinx and Altera chips.
On the basis of the simulation and the usage in a bigger design, it can be said that there are no known bugs.

\section{Timing}
\label{sec:timing}
Several tests were conducted to show the good performance regarding speed, one of the PicoBlazes key aspects.
The first test scenario is a simple I/O design based the PicoBlaze manual \cite[p.\,72]{KCPSM6} where one can also find the values used in this comparison.
The frequency of the PauloBlaze was increased step by step until the Xilinx tool \emph{trace} resulted in a timing violation. Table \ref{tab:timings} presents the measurements.
\begin{table}[h]
	\sffamily
	\captionof{table}{Achievable Speeds$^{1}$ with a simple I/O Design}
	\label{tab:timings}
	\sisetup{detect-family=true}
	\centering
	\begin{threeparttable}
	\begin{tabular}{lS[table-format=1]S[table-format=3,table-space-text-post = \si{\mega\hertz}]S[table-format=3,table-space-text-post = \si{\mega\hertz}]S[table-format=2.1,table-space-text-post = \si{\percent}]}
	\toprule
		Family		&	{Speed Grade}	&	{PicoBlaze}	&	{PauloBlaze}&	{Slowdown\tnote{2}}	\\	\midrule
		Kintex-7	&	-1			&	185\,\si{\mega\hertz}	&	156\,\si{\mega\hertz} 	&	15.7\,\si{\percent}		\\
					&	-3			&	238\,\si{\mega\hertz}	&	222\,\si{\mega\hertz} 		&	6.7\,\si{\percent}	\\	\midrule
		Virtex-7	&	-3			&	232\,\si{\mega\hertz}	&	222\,\si{\mega\hertz} 		&	4.3\,\si{\percent}	\\	\midrule
		Virtex-6	&	-3			&	238\,\si{\mega\hertz}	&	200\,\si{\mega\hertz} 		&	16.0\,\si{\percent}		\\	\midrule
		Spartan-6	&	-1L			&	82 \,\si{\mega\hertz}	&	74\,\si{\mega\hertz} 		&	9.8\,\si{\percent}		\\		
					&	-3			&	136\,\si{\mega\hertz}	&	114\,\si{\mega\hertz} 		&	16.2\,\si{\percent}		\\
					\bottomrule
	\end{tabular}
	\begin{tablenotes}
		\small
		\item[1] \si{\mega\hertz} values provided by the Xilinx tool \emph{trace} after place and route
		\item[2] slowdown of the PauloBlaze compared to the PicoBlaze
	\end{tablenotes}
	\end{threeparttable}
\end{table}
It is not surprising to get the same speed from two different 7-Series devices because they share the same architecture.
Also, this architecture and the good speed grade fit the implementation best.
On the other hand, a Spartan-6 with a low grade is less affected than one with a better grade.

Another test was conducted using the PicoBlaze-Library mentioned in section \ref{sec:verifi}.
It was augmented and deployed on the Atlys Board \cite{Atlys}, featuring a Spartan-6 (speed grad -3).
Both, PicoBlaze and PauloBlaze, were able to meet the timing requirements of \SI{100}{\mega\hertz} in this real world example.

\section{Resource Usage}
\label{sec:res_use}
Similar to the timing results the PauloBlaze utilizes more resources than its optimized, but restricted, counterpart as shown below with a few examples.
The first design was taken from the PicoBlaze Library and implemented on the same Atlys Board.
It contains a module called SoFPGA featuring extensions like a divider unit for the single processor.
The \emph{Pico constrained} represents the normal PicoBlaze instantiation whereas all the placement instructions were deleted in the \emph{unconstrained} version.
Without the directives the tools could place parts of the processor into other modules.
Such modules may also occupy a part of a slice, but the usage report assigns the whole slice to the processor.
Thus, the resulting utilization values are only estimates.
All values were taken after the Xilinx tool \emph{map} had finished.

\begin{figure}[h]
	\sffamily
	\centering
	\includegraphics[width=1.0\textwidth]{res_sofpga}
	\caption{Resource Usage of the surrounding SoFPGA Module including the processor, dividers, multipliers and more}
	\label{fig:res_sofpga}
\end{figure}

\begin{table}[h]
	\sffamily
	\centering
	\sisetup{detect-family=true}
	\caption{Resource Usage of the surrounding SoFPGA Module}
	\label{tab:res_sofpga}
	\begin{tabular}{l S[table-format=3] S[table-format=3] S[table-format=3] S[table-format=2.1,table-space-text-post = \si{\percent}] S[table-format=3.1,table-space-text-post = \si{\percent}]}
		\toprule
		Resource				&	{Pico unconstrained} & {Pico constrained} & {PauloBlaze} & \multicolumn{2}{c}{increase} \\ \midrule
		Slices			&	1266	&	1242	&	1348	& 6.5\,\si{\percent}	& 8.5\,\si{\percent}	\\	
		Slice Register	&	2924	&	2925	&	2928	& 0.1\,\si{\percent}	& 0.1\,\si{\percent}	\\	
		LUTs			&	2420	&	2402	&	2599	& 7.4\,\si{\percent}	& 8.2\,\si{\percent}	\\	
		LUTRAM			&	774		&	774		&	781		& 0.9\,\si{\percent}	& 0.9\,\si{\percent}	\\
		\bottomrule
	\end{tabular}
\end{table}

The differences between the two PicoBlaze versions seen in figure~\ref{fig:res_sofpga} and table~\ref{tab:res_sofpga} are most likely caused by the placers behaviour to spread out and use more of the chip when there is enough space left. It is not surprising to see an additional demand of resources because of the PauloBlaze. However, the increase, especially to the constrained version, is higher than expected compared to the following measurements.

On basis of the same utilization report a direct look at the single processor is possible. It needs to be pointed out that the report is less precise because of the small number of resources used and the shifting of those between different components.

\begin{figure}[h]
	\sffamily
	\centering
	\includegraphics[width=1.0\textwidth]{res_mod1}
	\caption{Resource Usage of an embedded Pico/PauloBlaze}
	\label{fig:res_mod}
\end{figure}

\begin{table}[H]
	\sffamily
	\centering
	\sisetup{detect-family=true}
	\caption{Resource Usage of an embedded Pico/PauloBlaze}	
	\label{tab:res_mod}
	\begin{tabular}{l S[table-format=3] S[table-format=3] S[table-format=3] S[table-format=2.1,table-space-text-post = \si{\percent}] S[table-format=3.1,table-space-text-post = \si{\percent}]}
		\toprule
		Resource			&	{Pico unconstrained} & {Pico constrained} & {PauloBlaze} & \multicolumn{2}{c}{increase} \\ \midrule
		Slices			&	57	&	33	&	100	& 75.4\,\si{\percent}	& 203.0\,\si{\percent}	\\	
		Slice Register	&	79	&	79	&	83	& 5.0\,\si{\percent}	& 5.1\,\si{\percent}	\\	
		LUTs			&	108	&	129	&	155	& 43.5\,\si{\percent}	& 20.2\,\si{\percent}	\\	
		LUTRAM			&	49	&	49	&	56	& 14.3\,\si{\percent}	& 14.3\,\si{\percent}	\\
		\bottomrule
	\end{tabular}
\end{table}

The shifting of resources from one module to another becomes clear in figure~\ref{fig:res_mod} when comparing the slice and LUT numbers of the two PicoBlaze versions.
On one hand the unconstrained version shares "its" slices with other components almost doubling the usage, on the other hand \SI{17}{\percent} of the LUTs are not counted.
The PauloBlaze shows an additional slice usage by 43 or 67.
This is less than the 100 extra slices suggested by the report of the SoFPGA module.

To get a precise and reasonable result, the timing sections I/O design was tested with different chip families.
It is a simple design with only a few extra constructs around the processor.
The normal constrained PicoPlaze was used and compared to the PauloBlaze.
\begin{table}[h]
	\sffamily
	\centering
	\sisetup{detect-family=true}
	\caption{Resource Usage of a Simple I/O Design}
	\label{tab:area}
	\begin{tabular}{l c S[table-format=3]S[table-format=3] c S[table-format=3]S[table-format=3] c S[table-format=3.1,table-space-text-post = \si{\percent}]S[table-format=3.1,table-space-text-post = \si{\percent}]}
		\toprule			
		\textbf{Family}		&	&	\multicolumn{2}{c}{\textbf{PicoBlaze}}	&&	\multicolumn{2}{c}{\textbf{PauloBlaze}}	&&	\multicolumn{2}{c}{\textbf{increase}}	\\	
					&&	{S. LUTs}	& {S. Reg}	&& {S. LUTs}	& {S. Reg}	&&	{S. LUTs}	& {S. Reg}	\\	\midrule
		Spartan-6	&&	114		&	84		&&	275		& 90		&&	141.2\,\si{\percent}	& 7.1\,\si{\percent}		\\
		Virtex-6	&&	121		&	115		&&	276		& 91		&&	128.1\,\si{\percent}	& -20.9\,\si{\percent}		\\
		Virtex-7	&&	113		&	83		&&	283		& 82		&&	150.4\,\si{\percent}	& -1.2\,\si{\percent}		\\		\bottomrule
	\end{tabular}
\end{table}
Although the numbers in table~\ref{tab:area} are pointing out a significant addition, it is a negligible increase regarding the overall utilization from \SI{1.7}{\percent} to \SI{3.1}{\percent} on a medium sized Spartan-6 (XC6SLX75).
