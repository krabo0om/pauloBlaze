\chapter{Introduction}
\label{ch:intro}

Field Programmable Gate Arrays (FPGAs) have always had a great potential for an efficient implementation of parallelizable algorithms.
Streaming based problems benefit from the low I/O latency and, thanks to the configurable cells, can be mapped very well.
However, step by step sequences have proven to be difficult. 
Complex tasks may demand a hybrid solution, configurable logic combined with a dedicated CPU on a single chip like a Xilinx Zynq or an Altera Cyclone.
In contrast state machines are suitable for short sequences and are often used.
But those state machines have practical limits, there is a huge bulk of states, one cannot cope with the number of transitions, more input values, more output signals, a more and more complex implementation leads to more resource usage.
This problem is not new and the typical solution is a so-called softcore.
Such a softcore contains the mapping of a CPU model onto the internal FPGA resources and the user can execute ordinary Assembler or C source code.
A popular core is the KCPSM6 also known as PicoBlaze \cite{PicoBlaze}. It is used in many applications like elliptic curve cryptography \cite{ref_elliptic}, a floating-point controller \cite{kadlec2005floating} or a multiprocessor system \cite{ref_paral_exec}.
Its implementation is utterly compact and with up to \SI{238}{\mega\hertz} very fast, however, because of its direct description of Look-Up Tables (LUTs) it is limited to current Xilinx FPGAs and not easily customizable.
\\ 
In this work a processor, called \textit{PauloBlaze}, is developed which is \SI{100}{\percent} compatible to the PicoBlazes ISA and all of the signal timings.
It should be very easy to replace a PicoBlaze in a current project, to modify this new implementation, or to deploy it without being restricted to particular platforms.
These benefits should outweigh the speed and area losses.
\\
This work is structured into four chapters.
An introduction and motivation is given in chapter~\ref{ch:intro} followed by chapter~\ref{ch:impl}, an overview on the implementation of the single modules and how they work together.
The resource usage is summarized in chapter~\ref{ch:res} and compared to the PicoBlazes.
Advantages and compromises of the PauloBlaze are discussed in the last chapter.
